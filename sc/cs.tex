\documentclass{article}

    \usepackage[utf8]{inputenc}
    \usepackage{polski}
    \usepackage{amsmath}
    \usepackage{graphicx} % rysuneczki
    \selecthyphenation{polish}

    \title{Ściąga}
    \author{PK}

\begin{document}

    \maketitle

\section{Good Moaning!}

    {\rm I shall} \textrm{say it only once.}

    {\bf I shall} \textbf{say it only once.}

    {\it I shall} \textit{say it only once.}

    {\tt I shall} \texttt{say it only once.}

    {\sl I shall} \textsl{say it only once.}

    {\sf I shall} \textsf{say it only once.}
    
    {\em I shall} \emph{ say it only once.}

    \textnormal{I shall say it only once.}

% \pagebreak
\section{Formuły}

    \begin{equation}
        E = mc^2 \;
        \label{eqn:clever}
    \end{equation}

    \begin{equation}
        \text{cov} (X,Y) = 
        \frac{\sum_{i=1}^n (x_i - \overline{x})(y_i - \overline{y})}
        {[\sum_{i=1}^n (x_i - \overline{x})^2 \sum_{i=1}^n (y_i - \overline{y})^2]^{1/2}}
    \end{equation}

    \begin{equation}
        F(a,b) = 
        \frac{\lim_{x \rightarrow \infty} \frac{x^3}{\sqrt[15]{1 - x} } + \sum_{i=1}^n \ln a_i}
        {\iint_0^{\infty} \; \sin (a) \, \cos (b) \mathrm{d}a \mathrm{d}b}
    \end{equation}

\section{Rysunki}

    \mbox{Tekst taki długi że aż się nie mieści w linijce, jeszce trochę w prawo, keszcze parę słów, jeszcze kawałek i nic nie widać}

    Rysunki robione przy pomocy \LaTeX{}-a

    \begin{figure}[h]
        \centering
        \setlength{\unitlength}{1cm}
        \begin{picture}(10,5)(0,0)
            \put( 0,0){\circle*{0.2}}
            \put(10,0){\circle*{0.2}}
            \put(10,5){\circle*{0.2}}
            \put( 0,5){\circle*{0.2}}

            \put(5,2.5){\circle{1.4}}
            \put(9,2.5){\vector(-1,0){3}}
            \put(6,2.8){$ \text{Koło o średnicy 14 mm} $}
            \thinlines
        \end{picture}
        \caption{
            Obrazek 10 cm na 5 cm.
        } 
    \end{figure}

    \begin{figure}[h]
        \centering
        \setlength{\unitlength}{1cm}
        \begin{picture}(6,3)(0,0)
            \put(0,0){\circle*{0.2}}
            \put(6,0){\circle*{0.2}}
            \put(6,3){\circle*{0.2}}
            \put(0,3){\circle*{0.2}}

            \put(1,1){\line(1,0){4}}
            \put(1,1){\line(4,1){4}}
            \put(5,1){\line(0,1){1}}
            \put(5.1,1){$ \text{A} $}
            \put(5.1,2){$ \text{B} $}
            \put(0.7,1){$ \text{C} $}
            \thinlines
        \end{picture}
        \caption{
            Obrazek 6 cm na 3 cm.
        } 
    \end{figure}

\section{Tabele}

    \listoftables   

    Tabele i spis tabel

    \begin{table}[h!]
        \caption[Niektóre stolice europejskie]{
            \label{tab.stolice}    
            Niektóre stolice europejskie. I rzeki też.
        }
        \centering
        \begin{tabular}{||l|l|l||}
            \hline\hline
            Panstwo   & Stolica  & Rzeka \\ \hline
            Polska    & Warszawa & Wisła \\
            Wlochy    & Rzym     & Tybr \\
            Francja   & Paryz    & Sekwana \\
            Finlandia & Helsinki & Amazonka \\
            \hline\hline
        \end{tabular}
    \end{table}

    \begin{table}[h!]
        \caption[Cechy morfologiczne kotowatych]{
            \label{tab.kotki}    
            Cechy morfologiczne wybranych gatunków kotowatych – cztery cechy
        }
        \centering
        \begin{tabular}{||l|c|c|c|c||}
            \hline\hline
            Kot & \multicolumn{4}{c|}{Cechy} \\ \hline
            Płeć & \multicolumn{2}{c|}{Samce} & \multicolumn{2}{c|}{Samice} \\ \hline
            Gatunek & Masa [kg] & Długosc [cm] & Masa [kg] & Długosc [cm] \\ \hline \hline
            Gepard & 53 & 230 & 38 & 220 \\ \hline
            Karakal & 16 & 100 & 12 & 90 \\ \hline
            Mormi & 15 & 150 & 10 & 120 \\ \hline\hline
        \end{tabular}
    \end{table}

\section{Cytowania}

Książka \cite{82_Ballard} i artykuł \cite{Lukac2005}.
I jeszcze jeden\cite{Bator2008}.

\bibliographystyle{alpha}
\bibliography{bibl} 

\end{document}