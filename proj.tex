\documentclass{article}

    \usepackage[utf8]{inputenc}
    \usepackage{polski}
    \selecthyphenation{polish}

    % \title{Chronicon Polonorum}
    % \author{Joannis de Czarnkow}

\begin{document}
\title{%
  Kronikaa \\
  \Large Jana z Czarnkowa \\
  \large archidyakona gnieźnieńskiego podkanclerzego \\ królestwa polskiego (1370-1384).} 

\maketitle

\pagebreak
\section{O śmierci Władysława Łokietka, króla polskiego}
Roku pańskiego 1333-go, w czwarty dzień Id Marcowych[1], zmarł przesławny pan Władysław, król polski, w zamku krakowskim, i tamże, w kościele krakowskim katedralnym, z lewej strony chóru, naprzeciw wielkiego ołtarza, spoczywa pogrzebany. Pozostawił on jedynego syna, imieniem Kazimierza, dziedzica królestwa i spadkobiercę, o którego podziwienia godnych czynach, wspaniałości i cnocie niżej się pisze. — Ten Kazimierz, jeszcze za życia ojca swego Władysława, a z jego rozkazu, dla dobra i pokoju królestwa polskiego, pojął za żonę córkę wielmożnego księcia pana Gedymina, wielkiego księcia litewskiego, imieniem Annę[2]; gdy ją zamierzał razem ze sobą koronować, matka jego, córka niegdyś Bolesława, księcia kaliskiego, syna Władysława Odonicza, powiedziała, że to podług prawa dziać się nie powinno, ponieważ żyje ona, prawa królowa koronowana, za jej życia przeto innej na to samo królestwo koronować nie należy[3]; wszelako, zgadzając się dobrotliwie na prośby syna, którego bardzo czule kochała, w stąpiła do klasztoru Ś. Klary w Starym Sączu, i tam, przyjąwszy regułę Ś. Franciszka, pod habitem zakonnym, razem z siostrami tam żyjącemi, do końca żywota służyła Panu z pobożnością i wielką pokorą[4].
  
\end{document}